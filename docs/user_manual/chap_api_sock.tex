\section{Socket calls}

% Short description/overview of module functions
With the socket calls, the user can open, close, bind, \ldots sockets and do read
or write operations. The provided transport protocols are UDP and TCP.

\subsection{pico$\_$socket$\_$open}

\subsubsection*{Description}
This function will be called to open a socket from the application level. The created
socket will be unbound.

\subsubsection*{Function prototype}
\begin{verbatim}
struct pico_socket *pico_socket_open(uint16_t net, uint16_t proto,
						void (*wakeup)(uint16_t ev, struct pico_socket *s));
\end{verbatim}
%struct pico$\_$socket *pico$\_$socket$\_$open(uint16$\_$t net, uint16$\_$t proto,
%						void (*wakeup)(uint16$\_$t ev, struct pico$\_$socket *s));

\subsubsection*{Parameters}
\begin{itemize}
\item net - Network protocol, PICO$\_$PROTO$\_$IPV4 = 0, PICO$\_$PROTO$\_$IPV6 = 41
\item proto - Transport protocol, PICO$\_$PROTO$\_$TCP = 6, PICO$\_$PROTO$\_$UDP = 17
\item wakeup - Callback function that accepts 2 parameters:
\begin{itemize}
\item ev - Events that apply to that specific socket, see further
\item s - Pointer to a socket of type struct pico$\_$socket
\end{itemize}
\end{itemize}

\subsubsection*{Possible events for sockets}
\begin{itemize}
\item PICO$\_$SOCK$\_$EV$\_$RD - trigerred when data arrived on the socket
\item PICO$\_$SOCK$\_$EV$\_$WR - trigerred when ready to write to the socket (TCP only)
\item PICO$\_$SOCK$\_$EV$\_$CONN - trigerred when connection is established (TCP only)
\item PICO$\_$SOCK$\_$EV$\_$CLOSE - trigerred when FIN packet received (TCP only)
\item PICO$\_$SOCK$\_$EV$\_$FIN - trigerred when the socket is closed (TCP only)
\item PICO$\_$SOCK$\_$EV$\_$ERR - trigerred when an error occurs
\end{itemize}

\subsubsection*{Return value}
On success, this call returns a pointer to the declared socket ("struct pico$\_$socket *").
On error the socket is not created, "NULL" is returned, and pico$\_$err is set appropriately.

\subsubsection*{Errors}
\begin{itemize}
\item PICO$\_$ERR$\_$EPROTONOSUPPORT - protocol not supported
\item PICO$\_$ERR$\_$ENETUNREACH - network unreachable 
\end{itemize}

\subsubsection*{Example}
\begin{verbatim}
sk_tcp = pico_socket_open(PICO_PROTO_IPV4, PICO_PROTO_TCP, &wakeup);
\end{verbatim}
%sk$\_$tcp = pico$\_$socket$\_$open(PICO$\_$PROTO$\_$IPV4, PICO$\_$PROTO$\_$TCP, $\&$wakeup);


\subsection{pico$\_$socket$\_$read}

\subsubsection*{Description}
This function will be called to read a string from a socket from the application level. The function checks whether or not the socket is bound.

\subsubsection*{Function prototype}
int pico$\_$socket$\_$read(struct pico$\_$socket *s, void *buf, int len);

\subsubsection*{Parameters}
\begin{itemize}
\item s - pointer to socket of type "struct pico$\_$socket"
\item buf - void pointer to the start of a string buffer where the string will be stored
\item len - length of the string that was read from the socket (in bytes)
\end{itemize}

\subsubsection*{Return value}
On success, this call returns an integer representing the number of bytes read. On error, -1 is returned, and pico$\_$err is set appropriately.

\subsubsection*{Errors}
\begin{itemize}
\item PICO$\_$ERR$\_$EINVAL - invalid argument
\item PICO$\_$ERR$\_$EIO - input/output error
\item PICO$\_$ERR$\_$ESHUTDOWN - cannot read after transport endpoint shutdown
\end{itemize}

\subsubsection*{Example}
bytesRead = pico$\_$socket$\_$read(sk$\_$tcp, buffer, bufferLength);


\subsection{pico$\_$socket$\_$write}

\subsubsection*{Description}
This function will be called to write a string to a socket from the application level.
This function also checks if the socket is bound, connected and that it isn't shutdown
locally. This is the preferred function to use when writing strings from application
level. 

\subsubsection*{Function prototype}
int pico$\_$socket$\_$write(struct pico$\_$socket *s, void *buf, int len);

\subsubsection*{Parameters}
\begin{itemize}
\item s - pointer to socket of type "struct pico$\_$socket"
\item buf - void pointer to the start of a string buffer where the string is stored
\item len - length of the string that is stored in the buffer (in bytes)
\end{itemize}

\subsubsection*{Return value}
On success, this call returns an integer representing the number of bytes written to the socket.
On error, -1 is returned, and pico$\_$err is set appropriately.

\subsubsection*{Errors}
\begin{itemize}
\item PICO$\_$ERR$\_$EINVAL - invalid argument
\item PICO$\_$ERR$\_$EIO - input/output error
\item PICO$\_$ERR$\_$ENOTCONN - the socket is not connected
\item PICO$\_$ERR$\_$ESHUTDOWN - cannot send after transport endpoint shutdown
\item PICO$\_$ERR$\_$EADDRNOTAVAIL - address not available
\item PICO$\_$ERR$\_$EHOSTUNREACH - host is unreachable
\item PICO$\_$ERR$\_$ENOMEM - not enough space
\item PICO$\_$ERR$\_$EAGAIN - resource temporarily unavailable
\end{itemize}

\subsubsection*{Example}
bytesWritten = pico$\_$socket$\_$write(sk$\_$tcp, buffer, bufLength);


\subsection{pico$\_$socket$\_$sendto}

\subsubsection*{Description}
This function is be called by the "pico$\_$socket$\_$write" and "pico$\_$socket$\_$send" functions.
This function sends a string from the local address to the remote address, without checking
if the remote is connected or not.

\subsubsection*{Function prototype}
int pico$\_$socket$\_$sendto(struct pico$\_$socket *s, void *buf, int len, void *dst, uint16$\_$t remote$\_$port);

\subsubsection*{Parameters}
\begin{itemize}
\item s - pointer to socket of type "struct pico$\_$socket"
\item buf - void pointer to the start of a string buffer where the string is stored
\item len - length of the string that is stored in the buffer (in bytes)
\item dst - pointer to the origin of the IPv4/IPv6 frame header
\item remote$\_$port - portnumber of the receiving socket
\end{itemize}

\subsubsection*{Return value}
On success, this call returns an integer representing the number of bytes written to the socket.
On error, -1 is returned, and pico$\_$err is set appropriately.

\subsubsection*{Errors}
\begin{itemize}
\item PICO$\_$ERR$\_$EADDRNOTAVAIL - address not available
\item PICO$\_$ERR$\_$EINVAL - invalid argument
\item PICO$\_$ERR$\_$EHOSTUNREACH - host is unreachable
\item PICO$\_$ERR$\_$ENOMEM - not enough space
\item PICO$\_$ERR$\_$EAGAIN - resource temporarily unavailable
\end{itemize}

\subsubsection*{Example}
bytesWritten = pico$\_$socket$\_$sendto(sk$\_$tcp, buf, len, $\&$sk$\_$tcp->remote$\_$addr, sk$\_$tcp->remote$\_$port);


\subsection{pico$\_$socket$\_$recvfrom}

\subsubsection*{Description}
This function is called to receive a string of data from the specified socket.
This function also checks if the socket is bound but not if it is connected or shutdown locally. 

\subsubsection*{Function prototype}
int pico$\_$socket$\_$recvfrom(struct pico$\_$socket *s, void *buf, int len, void *orig, uint16$\_$t *remote$\_$port);

\subsubsection*{Parameters}
\begin{itemize}
\item s - Pointer to socket of type "struct pico$\_$socket"
\item buf - Void pointer to the start of a string buffer where the string will be stored
\item len - Length of the string that will be stored in the buffer (in bytes)
\item orig - Pointer to the origin of the IPv4/IPv6 frame header
\item remote$\_$port - Portnumber of the sender socket 
\end{itemize}

\subsubsection*{Return value}
On success, this call returns an integer representing the number of bytes read from the socket.
On error, -1 is returned, and pico$\_$err is set appropriately.

\subsubsection*{Errors}
\begin{itemize}
\item PICO$\_$ERR$\_$EINVAL - invalid argument
\item PICO$\_$ERR$\_$ESHUTDOWN - cannot read after transport endpoint shutdown
\item PICO$\_$ERR$\_$EADDRNOTAVAIL - address not available
\end{itemize}

\subsubsection*{Example}
bytesRcvd = pico$\_$socket$\_$recvfrom(sk$\_$tcp, buf, bufLen, $\&$peer, $\&$port);


\subsection{pico$\_$socket$\_$send}

\subsubsection*{Description}
This function is called to send a string of data to the specified socket.
This function also checks if the socket is connected and then calls the
pico$\_$socket$\_$sendto function.

\subsubsection*{Function prototype}
int pico$\_$socket$\_$send(struct pico$\_$socket *s, void *buf, int len);

\subsubsection*{Parameters}
\begin{itemize}
\item s - Pointer to socket of type "struct pico$\_$socket"
\item buf - Void pointer to the start of a string buffer where the string is stored
\item len - Length of the string that is stored in the buffer (in bytes)
\end{itemize}

\subsubsection*{Return value}
On success, this call returns an integer representing the number of bytes written to
the socket. On error, -1 is returned, and pico$\_$err is set appropriately.

\subsubsection*{Errors}
\begin{itemize}
\item PICO$\_$ERR$\_$EINVAL - invalid argument
\item PICO$\_$ERR$\_$ENOTCONN - the socket is not connected
\item PICO$\_$ERR$\_$EADDRNOTAVAIL - address not available
\item PICO$\_$ERR$\_$EHOSTUNREACH - host is unreachable
\item PICO$\_$ERR$\_$ENOMEM - not enough space
\item PICO$\_$ERR$\_$EAGAIN - resource temporarily unavailable
\end{itemize}

\subsubsection*{Example}
bytesRcvd = pico$\_$socket$\_$send(sk$\_$tcp, buf, bufLen);


\subsection{pico$\_$socket$\_$recv}

\subsubsection*{Description}
This function directly calls the pico$\_$socket$\_$recvfrom function.

\subsubsection*{Function prototype}
int pico$\_$socket$\_$recv(struct pico$\_$socket *s, void *buf, int len);

\subsubsection*{Parameters}
\begin{itemize}
\item s - Pointer to socket of type "struct pico$\_$socket"
\item buf - Void pointer to the start of a string buffer where the string will be stored
\item len - Length of the string in the socket buffer (in bytes)
\end{itemize}

\subsubsection*{Return value}
On success, this call returns an integer representing the number of bytes read
from the socket. On error, -1 is returned, and pico$\_$err is set appropriately.

\subsubsection*{Errors}
\begin{itemize}
\item PICO$\_$ERR$\_$EINVAL - invalid argument
\item PICO$\_$ERR$\_$ESHUTDOWN - cannot read after transport endpoint shutdown
\item PICO$\_$ERR$\_$EADDRNOTAVAIL - address not available
\end{itemize}

\subsubsection*{Example}
bytesRcvd = pico$\_$socket$\_$recv(sk$\_$tcp, buf, bufLen);


\subsection{pico$\_$socket$\_$bind}

\subsubsection*{Description}
This function binds a local IP-address and port to the specified socket.

\subsubsection*{Function prototype}
int pico$\_$socket$\_$bind(struct pico$\_$socket *s, void *local$\_$addr, uint16$\_$t *port);

\subsubsection*{Parameters}
\begin{itemize}
\item s - Pointer to socket of type "struct pico$\_$socket"
\item local$\_$addr - Void pointer to the local IP-address
\item port - Local portnumber to bind with the socket
\end{itemize}

\subsubsection*{Return value}
On success, this call returns 0 after a succesfull bind.
On error, -1 is returned, and pico$\_$err is set appropriately.

\subsubsection*{Errors}
\begin{itemize}
\item PICO$\_$ERR$\_$EINVAL - invalid argument
\item PICO$\_$ERR$\_$ENXIO - no such device or address
\end{itemize}

\subsubsection*{Example}
errMsg = pico$\_$socket$\_$bind(sk$\_$tcp, $\&$sockaddr4->addr, $\&$sockaddr4->port);


\subsection{pico$\_$socket$\_$connect}

\subsubsection*{Description}
This function connects a local socket to a remote socket of a server that is listening.

\subsubsection*{Function prototype}
int pico$\_$socket$\_$connect(struct pico$\_$socket *s, void *srv$\_$addr, uint16$\_$t remote$\_$port);

\subsubsection*{Parameters}
\begin{itemize}
\item s - Pointer to socket of type "struct pico$\_$socket"
\item srv$\_$addr - Void pointer to the remote IP-address to connect to
\item remote$\_$port - Remote port number on which the socket will be connected to
\end{itemize} 

\subsubsection*{Return value}
On success, this call returns 0 after a succesfull connect.
On error, -1 is returned, and pico$\_$err is set appropriately.

\subsubsection*{Errors}
\begin{itemize}
\item PICO$\_$ERR$\_$EPROTONOSUPPORT - protocol not supported
\item PICO$\_$ERR$\_$EINVAL - invalid argument
\item PICO$\_$ERR$\_$EHOSTUNREACH - host is unreachable 
\end{itemize}

\subsubsection*{Example}
errMsg = pico$\_$socket$\_$connect(sk$\_$tcp, $\&$sockaddr4->addr, sockaddr4->port);


\subsection{pico$\_$socket$\_$listen}

\subsubsection*{Description}
A server can use this function when a socket is opened and bound to start listening to it.

\subsubsection*{Function prototype}
int pico$\_$socket$\_$listen(struct pico$\_$socket *s, int backlog);

\subsubsection*{Parameters}
\begin{itemize}
\item s - Pointer to socket of type "struct pico$\_$socket"
\item backlog - ???
\end{itemize}

\subsubsection*{Return value}
On success, this call returns 0 after a succesfull listen start.
On error, -1 is returned, and pico$\_$err is set appropriately. 

\subsubsection*{Errors}
\begin{itemize}
\item PICO$\_$ERR$\_$EINVAL - invalid argument
\item PICO$\_$ERR$\_$EISCONN - socket is connected
\end{itemize}

\subsubsection*{Example}
errMsg = pico$\_$socket$\_$listen(sk$\_$tcp, 3);


\subsection{pico$\_$socket$\_$accept}

\subsubsection*{Description}
When a server is listening on a socket and the client is trying to connect.
The server on his side will wakeup and acknowlegde the connection by calling the this function.

\subsubsection*{Function prototype}
struct pico$\_$socket *pico$\_$socket$\_$accept(struct pico$\_$socket *s, void *orig, uint16$\_$t *local$\_$port);

\subsubsection*{Parameters}
\begin{itemize}
\item s - Pointer to socket of type "struct pico$\_$socket"
\item orig - Pointer to the origin of the IPv4/IPv6 frame header
\item local$\_$port - Portnumber of the local socket
\end{itemize}

\subsubsection*{Return value}
On success, this call returns the pointer to a pico$\_$socket ("struct pico$\_$socket *") that
represents the client thas was just connected. On error, "NULL" is returned, and pico$\_$err
is set appropriately.

\subsubsection*{Errors}
\begin{itemize}
\item PICO$\_$ERR$\_$EINVAL - invalid argument
\item PICO$\_$ERR$\_$EAGAIN - resource temporarily unavailable
\end{itemize}

\subsubsection*{Example}
client = pico$\_$socket$\_$accept(sk$\_$tcp, $\&$peer, $\&$port);


\subsection{pico$\_$socket$\_$shutdown}

\subsubsection*{Description}
Used by the pico$\_$socket$\_$close function to shutdown read and write mode for
the specified socket. With this function one can close a socket for reading
and/or writing.

\subsubsection*{Function prototype}
int pico$\_$socket$\_$shutdown(struct pico$\_$socket *s, int mode);

\subsubsection*{Parameters}
\begin{itemize}
\item s - Pointer to socket of type "struct pico$\_$socket"
\item mode - PICO$\_$SHUT$\_$RDWR, PICO$\_$SHUT$\_$WR, PICO$\_$SHUT$\_$RD
\end{itemize}

\subsubsection*{Return value}
On success, this call returns 0 after a succesfull socket shutdown.
On error, -1 is returned, and pico$\_$err is set appropriately.

\subsubsection*{Errors}
\begin{itemize}
\item PICO$\_$ERR$\_$EINVAL - invalid argument
\end{itemize}

\subsubsection*{Example}
errMsg = pico$\_$socket$\_$shutdown(s, PICO$\_$SHUT$\_$RDWR);


\subsection{pico$\_$socket$\_$close}

\subsubsection*{Description}
Function used on application level to close a socket. Always closes read and write connection.

\subsubsection*{Function prototype}
int pico$\_$socket$\_$close(struct pico$\_$socket *s);

\subsubsection*{Parameters}
\begin{itemize}
\item s - Pointer to socket of type "struct pico$\_$socket"
\end{itemize}

\subsubsection*{Return value}
On success, this call returns 0 after a succesfull socket shutdown.
On error, -1 is returned, and pico$\_$err is set appropriately.

\subsubsection*{Errors}
\begin{itemize}
\item PICO$\_$ERR$\_$EINVAL - invalid argument
\end{itemize}

\subsubsection*{Example}
errMsg = pico$\_$socket$\_$close(sk$\_$tcp);