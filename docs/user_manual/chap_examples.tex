It is assumed that all examples include the appropriate header files
and a \textbf{main} routine that calls the \texttt{app$\_$x} functions to initialize
the example.

The most common header files are:
\begin{verbatim}
#include "pico_stack.h"
#include "pico_config.h"
#include "pico_dev_vde.h"
#include "pico_ipv4.h"
#include "pico_socket.h"
#include "pico_dev_tun.h"
#include "pico_nat.h"
#include "pico_icmp4.h"
#include "pico_dns_client.h"
#include "pico_dev_loop.h"
#include "pico_dhcp_client.h"
#include "pico_dhcp_server.h"
#include "pico_ipfilter.h"
\end{verbatim}

\section{Ping example}

\begin{verbatim}
#define NUM_PING 10

/* callback function for receiving ping reply */
void cb_ping(struct pico_icmp4_stats *s)
{
  char host[30];
  int time_sec = 0;
  int time_msec = 0;
  
  /* convert ip address from icmp4_stats structure to string */
  pico_ipv4_to_string(host, s->dst.addr);
  
  /* get time information from icmp4_stats structure */
  time_sec = s->time / 1000;
  time_msec = s->time % 1000;
  
  if (s->err == PICO_PING_ERR_REPLIED) {
  	/* print info if no error reported in icmp4_stats structure */
    dbg("%lu bytes from %s: icmp_req=%lu ttl=%lu time=%lu ms\n", \
    					s->size, host, s->seq, s->ttl, s->time);
    if (s->seq >= NUM_PING)
      exit(0);
  } else {
  	/* else, print error info */
    dbg("PING %lu to %s: Error %d\n", s->seq, host, s->err);
    exit(1);
  }
}

/* initialize the ping command */
void app_ping(char *dest)
{
  pico_icmp4_ping(dest, NUM_PING, 1000, 5000, 48, cb_ping);
}
\end{verbatim}


\section{UDP echo socket example}

\begin{verbatim}
struct pico_ip4 inaddr_any = { };

/* callback for UDP echo socket events */
void cb_udpecho(uint16_t ev, struct pico_socket *s)
{
  char recvbuf[1400];
  int read = 0;
  uint32_t peer;
  uint16_t port;

  /* process read event, data available */
  if (ev == PICO_SOCK_EV_RD) {
  	/* while data available in socket buffer, echo data to peer */
    do {
      read = pico_socket_recvfrom(s, recvbuf, 1400, &peer, &port);
      if (read > 0)
        pico_socket_sendto(s, recvbuf, r, &peer, port);
    } while(read > 0);
  }

  /* process error event, socket error occured */
  if (ev == PICO_SOCK_EV_ERR) {
    printf("Socket Error received. Bailing out.\n");
    exit(1);
  }

  printf("Received data from %08X:%u\n", peer, port);
}

/* initialize the UDP echo socket */
void app_udpecho(uint16_t source_port)
{
  struct pico_socket *s;
  uint16_t port_be = 0;
  
  /* set the source port for the socket */
  if (source_port == 0)
    port_be = short_be(5555);
  else
    port_be = short_be(source_port);

  /* open a UDP socket with the appropriate callback */
  s = pico_socket_open(PICO_PROTO_IPV4, PICO_PROTO_UDP, &cb_udpecho);
  if (!s)
    exit(1);

  /* bind the socket to port_be */
  if (pico_socket_bind(s, &inaddr_any, &port_be) != 0)
    exit(1);
}
\end{verbatim}


\section{TCP echo socket example}

\begin{verbatim}
#define BSIZE 1460

/* callback for TCP echo socket events */
void cb_tcpecho(uint16_t ev, struct pico_socket *s)
{
  char recvbuf[BSIZE];
  int read = 0, written = 0;
  int pos = 0, len = 0;
  struct pico_socket *sock_a;
  struct pico_ip4 orig;
  uint16_t port;
  char peer[30];

  /* process read event, data available */
  if (ev & PICO_SOCK_EV_RD) {
    do {
      read = pico_socket_read(s, recvbuf + len, BSIZE - len);
      if (read > 0)
        len += read;
    } while(read > 0);
  }
  
  /* process connect event, syn received */
  if (ev & PICO_SOCK_EV_CONN) {
    /* accept new connection request */
    sock_a = pico_socket_accept(s, &orig, &port);
    
   	/* convert peer IP to string */
    pico_ipv4_to_string(peer, orig.addr);
    
    /* print info */
    printf("Connection established with %s:%d.\n", peer, short_be(port));
  }

  /* process fin event, receiving socket closed */
  if (ev & PICO_SOCK_EV_FIN) {
    printf("Socket closed. Exit normally. \n");
  }

  /* process error event, socket error occured */
  if (ev & PICO_SOCK_EV_ERR) {
    printf("Socket Error received: %s. Bailing out.\n", strerror(pico_err));
    exit(1);
  }
  
  /* process close event, receiving socket received close from peer */
  if (ev & PICO_SOCK_EV_CLOSE) {
    printf("Socket received close from peer.\n");
    /* shutdown write side of socket */
    pico_socket_shutdown(s, PICO_SHUT_WR);
  }

  /* if data read, echo back */
  if (len > pos) {
    do {
      /* echo data back to peer */
      written = pico_socket_write(s, recvbuf + pos, len - pos);
      if (written > 0) {
        pos += written;
        if (pos >= len) {
          pos = 0;
          len = 0;
          written = 0;
        }
      } else {
        printf("SOCKET> ECHO write failed, dropped %d bytes\n",(len-pos));
      }
    } while(written > 0);
  }
}

/* initialize the TCP echo socket */
void app_tcpecho(uint16_t source_port)
{
  struct pico_socket *s;
  uint16_t port_be = 0;
  int backlog = 40;			/* max number of accepting connections */
  int ret;
  
  /* set the source port for the socket */
  if (source_port == 0)
    port_be = short_be(5555);
  else
    port_be = short_be(source_port);

  /* open a TCP socket with the appropriate callback */
  s = pico_socket_open(PICO_PROTO_IPV4, PICO_PROTO_TCP, &cb_tcpecho);
  if (!s)
    exit(1);

  /* bind the socket to port_be */
  ret = pico_socket_bind(s, &inaddr_any, &port_be);
  if (ret != 0)
    exit(1);

  /* start listening on socket */
  ret = pico_socket_listen(s, backlog);
  if (ret != 0)
    exit(1);
}
\end{verbatim}


\section{NAT setup example}

\begin{verbatim}
/* initialize NAT functionality and add port forward rule */
void app_nat(char *dest)
{
  char *dest = NULL;
  struct pico_ip4 ipdst, pub_addr, priv_addr;
  struct pico_ipv4_link *link;

  /* convert IP address of link where to enable NAT */
  pico_string_to_ipv4(dest, &ipdst.addr);
  
  /* get link pointer */
  link = pico_ipv4_link_get(&ipdst);
  if (!link) {
    printf("destination not found\n");
    exit(1);
  }
  
  /* enable NAT on link */
  pico_ipv4_nat_enable(link);
  
  /* add port forward rule */
  pico_string_to_ipv4("10.50.0.10", &pub_addr.addr);
  pico_string_to_ipv4("10.40.0.08", &priv_addr.addr);
  pico_ipv4_port_forward(pub_addr, short_be(5555), priv_addr, short_be(6667),
  PICO_PROTO_UDP, PICO_IPV4_FORWARD_ADD);
  
  printf("nat started\n");
}
\end{verbatim}


\section{DNS example}

\begin{verbatim}
/* identifier struct */
struct dns_identifier {
  uint8_t id;
  /* ... */
};

/* callback function of URL translation */
void cb_getaddr(char *ip, void *arg)
{
  struct dns_identifier *id_getaddr = (struct dns_identifier *) arg;
  
  /* NULL indicates an error condition */
  if (!ip) {
    printf("DNS error occured: %s\n", strerror(pico_err));
    return;
  }
  printf("DNS translation to ip %s (id %u)\n", ip, id_getaddr ? id_getaddr->id : 0);
  
  /* important: free the received pointers! */
  PICO_FREE(ip);
  if (id_getaddr)
    PICO_FREE(id_getaddr);
}

/* callback function of IP translation */
void cb_getname(char *url)
{
  struct dns_identifier *id_getname = (struct dns_identifier *) arg;
  
  /* NULL indicates an error condition */
  if (!url) {
    printf("DNS error occured: %s\n", strerror(pico_err));
    return;
  }
  printf("DNS translation to url %s (id %u)\n", url, id_getname ? id_getname->id : 0);
  
  /* important: free the received pointers! */
  PICO_FREE(url);
  if (id_getname)
    PICO_FREE(id_getname);
}

/* initialize the dns */
void app_dns(char *url, char *ip)
{
  struct pico_ip4 nameserver = { };
  struct dns_identifier *id_getaddr = NULL, *id_getname = NULL;
  
  /* optional: add custom dns nameserver */
  pico_string_to_ipv4("8.8.4.4", &nameserver.addr);
  pico_dns_client_nameserver(&nameserver, PICO_DNS_NS_ADD);

  /* request translation of URL f.e. www.google.com */  
  id_getaddr = PICO_ZALLOC(sizeof(struct dns_identifier));
  id_getaddr->id = 1;
  pico_dns_client_getaddr(url, &cb_getaddr, id_getaddr);

  /* request translation of IP f.e. 8.8.8.8 */
  id_getname = PICO_ZALLOC(sizeof(struct dns_identifier));
  id_getname->id = 2;
  pico_dns_client_getname(ip, &cb_getname, id_getname);
}
\end{verbatim}


\section{DHCP client example}

\begin{verbatim}
int main(void)
{
  uint8_t mac_eth0[6] = {0x12, 0x34, 0x56, 0x78, 0x9a, 0xbc};
  uint8_t mac_eth1[6] = {0xcb, 0xa9, 0x87, 0x65, 0x43, 0x21};
  char s_addr_eth0[16] = { }, s_addr_eth1[16] = { };
  void *identifier_eth0 = NULL, *identifier_eth1 = NULL;
  uint32_t xid_eth0 = 0, xid_eth1 = 0;
  struct pico_device *eth0 = NULL, *eth1 = NULL;
  struct pico_ip4 addr_eth0 = { }, addr_eth1 = { };

  /* see section 2.5 Network devices integration */
  eth0 = pico_device_create("eth0", mac_eth0);
  eth1 = pico_device_create("eth1", mac_eth1);
  
  pico_stack_init();
  
  if (pico_dhcp_initiate_negotiation(eth0, &cb_dhcpclient, &xid_eth0) < 0) {
      printf("DHCPC: error initiating negotiation: %s\n", strerror(pico_err));
      exit(255);
  } 
  if (pico_dhcp_initiate_negotiation(eth1, &cb_dhcpclient, &xid_eth1) < 0) {
      printf("DHCPC: error initiating negotiation: %s\n", strerror(pico_err));
      exit(255);
  }
  
  for(;;) {
    pico_stack_tick();
    /* did both devices get a successful lease? */
    if (xid_eth0 && xid_eth1)
      break;
    PICO_IDLE();
  }
  
  identifier_eth0 = pico_dhcp_get_identifier(xid_eth0);
  addr_eth0 = pico_dhcp_get_address(identifier_eth0);
  pico_ipv4_to_string(s_addr_eth0, addr_eth0.addr);
  printf("Device %s got leased IP %s\n", eth0->name, s_addr_eth0);
  
  identifier_eth1 = pico_dhcp_get_identifier(xid_eth1);
  addr_eth1 = pico_dhcp_get_address(identifier_eth1);
  pico_ipv4_to_string(s_addr_eth1, addr_eth1.addr);
  printf("Device %s got leased IP %s\n", eth1->name, s_addr_eth1);
  
  return 0;
}
\end{verbatim}

\section{HTTP Client example}
\begin{verbatim}
static char *url_filename = NULL;

static int http_save_file(void *data, int len)
{
  int fd = open(url_filename, O_WRONLY |O_CREAT | O_TRUNC, 0660);
  int w, e;
  if (fd < 0)
    return fd;

  printf("Saving data to : %s\n",url_filename);
  w = write(fd, data, len);
  e = errno;
  close(fd);
  errno = e;
  return w;
}
void wget_callback(uint16_t ev, uint16_t conn)
{
  char data[1024 * 1024]; // MAX: 1M
  static int _length = 0;

  if(ev & EV_HTTP_CON)
  {
    printf(">>> Connected to the client \n");
    /* you can let the client use the default generated header
       or you can create you own string header (compatible with HTTP/1.x */
    pico_http_client_sendHeader(conn,NULL,HTTP_HEADER_DEFAULT);
  }

  if(ev & EV_HTTP_REQ)
  {
    struct pico_http_header * header = pico_http_client_readHeader(conn);
    printf("Received header from server...\n");
    printf("Server response : %d\n",header->responseCode);
    printf("Location : %s\n",header->location);
    printf("Transfer-Encoding : %d\n",header->transferCoding);
    printf("Size/Chunk : %d\n",header->contentLengthOrChunk);
  }

  if(ev & EV_HTTP_BODY)
  {
    int len;

    printf("Reading data...\n");
    /*
      Data is passed to you without you worrying if the transfer is 
      chunked or the content-length was specified.
    */
    while((len = pico_http_client_readData(conn,data + _length,1024)))
    {
      _length += len;
    }
  }

  if(ev & EV_HTTP_CLOSE)
  {
    struct pico_http_header * header = pico_http_client_readHeader(conn);
    int len;
    printf("Connection was closed...\n");
    printf("Reading remaining data, if any ...\n");
    while((len = pico_http_client_readData(conn,data,1000u)) && len > 0)
    {
      _length += len;
    }
    printf("Read a total data of : %d bytes \n",_length);

    if(header->transferCoding == HTTP_TRANSFER_CHUNKED)
    {
      if(header->contentLengthOrChunk)
      {
        printf("Last chunk data not fully read !\n");
        exit(1);
      }
      else
      {
        printf("Transfer ended with a zero chunk! OK !\n");
      }
    } else
    {
      if(header->contentLengthOrChunk == _length)
      {
        printf("Received the full : %d \n",_length);
      }
      else
      {
        printf("Received %d , waiting for %d\n",_length, header->contentLengthOrChunk);
        exit(1);
      }
    }

    if (!url_filename) {
      printf("Failed to get local filename\n");
      exit(1);
    }

    if (http_save_file(data, _length) < _length) {
      printf("Failed to save file: %s\n", strerror(errno));
      exit(1);
    }
    pico_http_client_close(conn);
    exit(0);
  }

  if(ev & EV_HTTP_ERROR)
  {
    printf("Connection error (probably dns failed : check the routing table), trying to close the client...\n");
    pico_http_client_close(conn);
    exit(1u);
  }

  if(ev & EV_HTTP_DNS)
  {
    printf("The DNS query was successful ... \n");
  }
}

void app_wget(char *arg)
{
  char * url;
  cpy_arg(&url, arg);

  if(!url)
  {
    fprintf(stderr, " wget expects the url to be received\n");
    exit(1);
  }
  
  // when opening the http client it will internally parse the url passed
  if(pico_http_client_open(url,wget_callback) < 0)
  {
    fprintf(stderr," error opening the url : %s, please check the format\n",url);
    exit(1);
  }
  url_filename = basename(url);
}
\end{verbatim}

\section{HTTP Server example}
\begin{verbatim}

#define SIZE 4*1024

void serverWakeup(uint16_t ev,uint16_t conn)
{
  static FILE * f;
  char buffer[SIZE];

  if(ev & EV_HTTP_CON)
  {
      printf("New connection received....\n");
      pico_http_server_accept();
  }

  if(ev & EV_HTTP_REQ) // new header received
  {
    int read;
    char * resource;
    printf("Header request was received...\n");
    printf("> Resource : %s\n",pico_http_getResource(conn));
    resource = pico_http_getResource(conn);

    if(strcmp(resource,"/")==0 || strcmp(resource,"index.html") == 0 || strcmp(resource,"/index.html") == 0)
    {
          // Accepting request
          printf("Accepted connection...\n");
          pico_http_respond(conn,HTTP_RESOURCE_FOUND);
          f = fopen("test/examples/index.html","r");

          if(!f)
          {
            fprintf(stderr,"Unable to open the file /test/examples/index.html\n");
            exit(1);
          }

          read = fread(buffer,1,SIZE,f);
          pico_http_submitData(conn,buffer,read);
    }
    else
    { // reject
      printf("Rejected connection...\n");
      pico_http_respond(conn,HTTP_RESOURCE_NOT_FOUND);
    }

  }

  if(ev & EV_HTTP_PROGRESS) // submitted data was sent
  {
    uint16_t sent, total;
    pico_http_getProgress(conn,&sent,&total);
    printf("Chunk statistics : %d/%d sent\n",sent,total);
  }

  if(ev & EV_HTTP_SENT) // submitted data was fully sent
  {
    int read;
    read = fread(buffer,1,SIZE,f);
    printf("Chunk was sent...\n");
    if(read > 0)
    {
        printf("Sending another chunk...\n");
        pico_http_submitData(conn,buffer,read);
    }
    else
    {
        printf("Last chunk !\n");
        pico_http_submitData(conn,NULL,0);// send the final chunk
        fclose(f);
    }
  }

  if(ev & EV_HTTP_CLOSE)
  {
    printf("Close request...\n");
    pico_http_close(conn);
  }

  if(ev & EV_HTTP_ERROR)
  {
    printf("Error on server...\n");
    pico_http_close(conn);
  }
}
/* simple server example that serves the index(.html) page */
void app_httpd(char *arg)
{
  /* transfer encoding with this server is always chunked and you can 
     submit chunks to the client, without needing to specify the content-length of the
     body response */ 
  if( pico_http_server_start(0,serverWakeup) < 0)
  {
    fprintf(stderr,"Unable to start the server on port 80\n");
  }
}
\end{verbatim}

